\documentclass[12pt]{article}
\usepackage{amsfonts,amssymb,amsmath}

\input{mathmac}
\input{mac-new}
\input{mathmac-v2}

\setlength{\topmargin}{-.5in}
\setlength{\oddsidemargin}{0 in}
\setlength{\evensidemargin}{0 in}
\setlength{\textwidth}{6.5truein}
\setlength{\textheight}{8.5truein}


\def\fseq#1#2{(#1_{#2})_{#2\geq 1}}
\def\fsseq#1#2#3{(#1_{#3(#2)})_{#2\geq 1}}
\def\qleq{\sqsubseteq}

\begin{document}
\begin{center}

{\Large\bf Module 2\\
\vspace{0.5cm}
Matrices and Linear Maps}\\[10pt]

\end{center}

\section*{Problem 1: 20 points total}
\label{prob-3.2}
\begin{itemize}
\item[(1)](5 points)
Prove that the column vectors of the matrix $A_2$ given by 
\[
A_2 = 
\begin{pmatrix}
1 & 1 & 1 & 1\\
1 & 2 & 1 & 3 \\
1 & 1 & 2 & 2 \\
1 & 1 & 1 & 3
\end{pmatrix}
\]
are linearly  independent. \\

We can prove linear independence by solving a linear system of equations such that the only solution for $a$, $b$, $c$, $d$ is 0. \\
\begin{align*}
    a * \begin{pmatrix}
    1 \\
    1 \\
    1 \\
    1 \\
    \end{pmatrix} + b * \begin{pmatrix}
    1 \\
    2 \\
    1 \\
    1 \\
    \end{pmatrix} + c * \begin{pmatrix}
    1 \\
    1 \\
    2 \\
    1 \\
    \end{pmatrix} + d * \begin{pmatrix}
    1 \\
    3 \\
    2 \\
    3 \\
    \end{pmatrix} &= \begin{pmatrix}
    0 \\
    0 \\
    0 \\
    0 \\
    \end{pmatrix}
\end{align*}

\begin{align*}
    a + b + c + d &= 0 \label{eq:one} \tag{1} \\
    a + 2b + c + 3d &= 0 \label{eq:two} \tag{2}\\
    a + b + 2c + 2d &= 0 \label{eq:three} \tag{3} \\
    a + b + c + 3d &= 0 \label{eq:four} \tag{4} \\
\end{align*}

Subtracting \eqref{eq:one} from \eqref{eq:four} we get: 
\[3d = 0\]

Therefore, $d=0$. Subtracting \eqref{eq:two} from \eqref{eq:four}, we get:

\[b = 0\]

We now know that $b=0$. Using \eqref{eq:one},
\begin{align*}
    a + b + c + d &= 0 \\
    a &= -b - c - d \\
\end{align*}

Substituing this into \eqref{eq:three}, we get:
\begin{align*}
    -b - c - d + + b + 2c + 2d &= 0 \\
    c + d &= 0 \\
    c + 0 &= 0 \\
    c &= 0 \\
\end{align*}

Finally, we can use \eqref{eq:two} to solve for $a$:
\begin{align*}
    a + 2b + c + 3d &= 0 \\
    a + 2 * 0 + 0 + 3 * 0 &= 0 \\
    a &= 0 \\
\end{align*}

Thus, we have proven that the only solution to the linear system of equations is 0, proving this matrix's columns are linearly independent.

\item[(2)](5 points)
Prove that the column vectors of the matrix $B_2$ given by 
\[
B_2 = 
\begin{pmatrix}
1 & -2 & 2 & -2\\
0 & -3 & 2 & -3 \\
3 & -5 & 5 & -4 \\
3 & -4 & 4 & -4
\end{pmatrix}
\]
are linearly  independent.

We can prove linear independence by solving a linear system of equations such that the only solution for $a$, $b$, $c$, $d$ is 0. \\
\begin{align*}
    a * \begin{pmatrix}
    1 \\
    0 \\
    3 \\
    3 \\
    \end{pmatrix} + b * \begin{pmatrix}
    -2 \\
    -3 \\
    -5 \\
    -4 \\
    \end{pmatrix} + c * \begin{pmatrix}
    2 \\
    2 \\
    5 \\
    4 \\
    \end{pmatrix} + d * \begin{pmatrix}
    -2 \\
    -3 \\
    -4 \\
    -4 \\
    \end{pmatrix} &= \begin{pmatrix}
    0 \\
    0 \\
    0 \\
    0 \\
    \end{pmatrix}
\end{align*}

\begin{align*}
    a - 2b + 2c - 2d &= 0 \label{eq:one} \tag{1} \\
    -3b + 2c - 3d &= 0 \label{eq:two} \tag{2}\\
    3a - 5b + 5c - 4d &= 0 \label{eq:three} \tag{3} \\
    3a - 4b + 4c - 4d &= 0 \label{eq:four} \tag{4} \\
\end{align*}

Using \eqref{eq:one}, we can solve for $a$ as follows:
\begin{align*}
    a - 2b + 2c - 2d &= 0 \\
    a = 2b - 2c + 2d \\
\end{align*}

We can substitute this into \eqref{eq:three}:
\begin{align*}
    3a - 5b + 5c - 4d &= 0 \\
    3 * (2b - 2c + 2d) - 5b + 5c - 4d &= 0 \\
    6b - 6c + 6d - 5b + 5c - 4d &= 0 \\
    b - c + 2d &= 0 \\
    b &= c - 2d \\
\end{align*}

Using \eqref{eq:four}, we can substitute $a$ and $b$ as follows: 
\begin{align*}
    3a - 4b + 4c - 4d &= 0 \\
    3 * (2b - 2c + 2d) - 4 * (c - 2d) + 4c - 4d &= 0 \\
    3 *(2 * (c - 2d) - 2c + 2d) - 4c + 8d + 4c - 4d &= 0 \\
    3 * (2c - 4d - 2c + 2d) + 4d &= 0 \\
    6c - 12d - 6c + 6d + 4d &= 0 \\
    -2d &= 0 \\
    d &= 0 \\
\end{align*}

Plugging this into \eqref{eq:two}, 
\begin{align*}
    -3b + 2c - 3d &= 0 \\
    -3 * (c - 2d) + 2c - 3 * 0 &= 0 \\
    -3c + 6d + 2c &= 0 \\
    -c &= 0 \\
    c &= 0 \\
\end{align*}

This implies that $b = c - 2d = 0 - 2 * 0 = 0$. This implies from the first solving for $a$: $a = 2b - 2c + 2d = 2 * 0 - 2 * 0 + 2 * 0 = 0$. Therefore we can say that the columns of $B_2$ are linearly independent.

\item[(3)](10 points)
Prove that the coordinates of the column vectors of the matrix $B_2$
over the basis consisting of the column vectors of $A_2$ 
are the columns of the matrix $P_2$ given by
\[
P_2 = 
\begin{pmatrix}
2 & 0 & 1 & -1\\
-3 & 1 & -2 & 1 \\
1 & -2 & 2 & -1 \\
1 & -1 & 1 & -1
\end{pmatrix}
\].
Check that $A_2P_2 = B_2$. 

We need to solve: 
\begin{align*}
    \begin{pmatrix}
    1 & 1 & 1 & 1\\
    1 & 2 & 1 & 3 \\
    1 & 1 & 2 & 2 \\
    1 & 1 & 1 & 3
    \end{pmatrix} *
    \begin{pmatrix}
    2 & 0 & 1 & -1\\
    -3 & 1 & -2 & 1 \\
    1 & -2 & 2 & -1 \\
    1 & -1 & 1 & -1
    \end{pmatrix} 
    &=  \begin{pmatrix}
    1 & -2 & 2 & -2\\
    0 & -3 & 2 & -3 \\
    3 & -5 & 5 & -4 \\
    3 & -4 & 4 & -4 
    \end{pmatrix}
\end{align*}
We can simplify this to be $a * p = c$, where the subscript represents, the row and then the column. We want to show that $b = c$.
\begin{align*}
    c_{1,1} &= 1 * 2 + 1 * -3 + 1 * 1 + 1 * 1 \\
    c_{1,1} &= 2 - 3 + 1 + 1 \\
    c_{1,1} &= 1 \\
\end{align*}

\begin{align*}
    c_{1,2} &= 1 * 0 + 1 * 1 + 1 * -2 + 1 * -1 \\
    c_{1,2} &= 0 + 1 - 2 - 1 \\
    c_{1,2} &= -2 \\
\end{align*}

\begin{align*}
    c_{1,3} &= 1 * 1 + 1 * -2 + 1 * 2 + 1 * 1 \\
    c_{1,3} &= 1 - 2 + 2 + 1 \\
    c_{1,3} &= 2 \\
\end{align*}

\begin{align*}
    c_{1,4} &= 1 * -1 + 1 * 1 + 1 * -1 + 1 * -1 \\
    c_{1,4} &= -1 + 1 - 1 - 1 \\
    c_{1,4} &= -2 \\
\end{align*}

\begin{align*}
    c_{2,1} &= 1 * 2 + 2 * -3 + 1 * 1 + 3 * 1 \\
    c_{2,1} &= 2 - 6 + 1 + 3 \\
    c_{2,1} &= 0 \\
\end{align*}

\begin{align*}
    c_{2,2} &= 1 * 0 + 2 * 1 + 1 * -2 + 3 * -1 \\
    c_{2,2} &= 0 + 2 - 2 - 3 \\
    c_{2,2} &= -3 \\
\end{align*}

\begin{align*}
    c_{2,3} &= 1 * 1 + 2 * -2 + 1 * 2 + 3 * 1 \\
    c_{2,3} &= 1 - 4 + 2 + 3 \\
    c_{2,3} &= 2 \\
\end{align*}

\begin{align*}
    c_{2,4} &= 1 * -1 + 2 * 1 + 1 * -1 + 3 * -1 \\
    c_{2,4} &= -1 + 2 - 1 - 3 \\
    c_{2,4} &= -3 \\
\end{align*}

\begin{align*}
    c_{3,1} &= 1 * 2 + 1 * -3 + 2 * 1 + 2 * 1 \\
    c_{3,1} &= 2 - 3 + 2 + 2 \\
    c_{3,1} &= 3  \\
\end{align*}

\begin{align*}
    c_{3,2} &= 1 * 0 + 1 * 1 + 2 * -2 + 2 * -1 \\
    c_{3,2} &= 0 + 1 - 4 - 2 \\
    c_{3,2} &= -5  \\
\end{align*}

\begin{align*}
    c_{3,3} &= 1 * 1 + 1 * -2 + 2 * 2 + 2 * 1 \\
    c_{3,3} &= 1 - 2 + 4 + 2 \\
    c_{3,3} &= 5  \\
\end{align*}

\begin{align*}
    c_{3,4} &= 1 * -1 + 1 * 1 + 2 * -1 + 2 * -1 \\
    c_{3,4} &= -1 + 1 - 2 - 2 \\
    c_{3,4} &= -4  \\
\end{align*}

\begin{align*}
    c_{4,1} &= 1 * 2 + 1 * -3 + 1 * 1 + 3 * 1 \\
    c_{4,1} &= 2 - 3 + 1 + 3 \\
    c_{4,1} &= 3  \\
\end{align*}

\begin{align*}
    c_{4,2} &= 1 * 0 + 1 * 1 + 1 * -2 + 3 * -1 \\
    c_{4,2} &= 0 + 1 - 2 - 3 \\
    c_{4,2} &= -4  \\
\end{align*}

\begin{align*}
    c_{4,3} &= 1 * 1 + 1 * -2 + 1 * 2 + 3 * 1 \\
    c_{4,3} &= 1 - 2 + 2 + 3 \\
    c_{4,3} &= 4  \\
\end{align*}

\begin{align*}
    c_{4,4} &= 1 * -1 + 1 * 1 + 1 * -1 + 3 * -1 \\
    c_{4,4} &= -1 + 1 - 1 - 3 \\
    c_{4,4} &= -4  \\
\end{align*}

Thus we have proven that the coordinates of the column vectors of the matrix $B_2$ over the basis consisting of the column vectors of $A_2$ are the columns of the matrix $P_2$ since $b = c$. \\

Prove that
\[
P_2^{-1} = 
\begin{pmatrix}
-1 & -1 & -1 & 1\\
2 & 1 & 1 & -2 \\
2 & 1 & 2 & -3 \\
-1 & -1 & 0 & -1
\end{pmatrix} .
\]

To prove that this is the inverse of $P_2$, we can multipy the two matrices together and if they equal the identity matrix, we can say that this is indeed the inverse of $P_2$.

\begin{align*}
    \begin{pmatrix}
    2 & 0 & 1 & -1\\
    -3 & 1 & -2 & 1 \\
    1 & -2 & 2 & -1 \\
    1 & -1 & 1 & -1
    \end{pmatrix} *
    \begin{pmatrix}
    -1 & -1 & -1 & 1\\
    2 & 1 & 1 & -2 \\
    2 & 1 & 2 & -3 \\
    -1 & -1 & 0 & -1
    \end{pmatrix} &= 
    \begin{pmatrix}
    1 & 0 & 0 & 0 \\
    0 & 1 & 0 & 0 \\
    0 & 0 & 1 & 0 \\
    0 & 0 & 0 & 1
    \end{pmatrix}
\end{align*}

We will solve for the matrix, $c$ as we did before and then will check to see if $c$ is equal to $I$. 

\begin{align*}
    c_{1,1} &= 2 * -1 + 0 * 2 + 1 * 2 + -1 * -1 \\
    c_{1,1} &= -2 + 0 + 2 + 1 \\
    c_{1,1} &= 1 \\
\end{align*}

\begin{align*}
    c_{1,2} &= 2 * -1 + 0 * 1 + 1 * 1 + -1 * -1 \\
    c_{1,2} &= -2 + 0 + 1 + 1 \\
    c_{1,2} &= 2 \\
\end{align*}

\begin{align*}
    c_{1,3} &= 2 * -1 + 0 * 1 + 1 * 2 + -1 * 0 \\
    c_{1,3} &= -2 + 0 + 2 + 0 \\
    c_{1,3} &= 0 \\
\end{align*}

\begin{align*}
    c_{1,4} &= 2 * 1 + 0 * -2 + 1 * -3 + -1 * -1 \\
    c_{1,4} &= 2 + 0 - 3 + 1 \\
    c_{1,4} &= 0 \\
\end{align*}

\begin{align*}
    c_{2,1} &= -3 * -1 + 1 * 2 + -2 * 2 + 1 * -1 \\
    c_{2,1} &= 3 + 2 - 4 - 1 \\
    c_{2,1} &= 0 \\
\end{align*}

\begin{align*}
    c_{2,2} &= -3 * -1 + 1 * 1 + -2 * 1 + 1 * -1 \\
    c_{2,2} &= 3 + 1 - 2 - 1 \\
    c_{2,2} &= 1 \\
\end{align*}

\begin{align*}
    c_{2,3} &= -3 * -1 + 1 * 1 + -2 * 2 + 1 * 0 \\
    c_{2,3} &= 3 + 1 - 4 + 0 \\
    c_{2,3} &= 0 \\
\end{align*}

\begin{align*}
    c_{2,4} &= -3 * 1 + 1 * -2 + -2 * -3 + 1 * -1 \\
    c_{2,4} &= -3 - 2 + 6 - 1 \\
    c_{2,4} &= 0 \\
\end{align*}

\begin{align*}
    c_{3,1} &= 1 * -1 + -2 * 2 + 2 * 2 + -1 * -1 \\
    c_{3,1} &= -1 - 4 + 4 + 1 \\
    c_{3,1} &= 0  \\
\end{align*}

\begin{align*}
    c_{3,2} &= 1 * -1 + -2 * 1 + 2 * 2 + -1 * -1 \\
    c_{3,2} &= -1 - 2 + 4 + 1 \\
    c_{3,2} &= 0  \\
\end{align*}

\begin{align*}
    c_{3,3} &= 1 * -1 + -2 * 1 + 2 * 2 + -1 * 0 \\
    c_{3,3} &= -1 - 2 + 4 + 0 \\
    c_{3,3} &= 1  \\
\end{align*}

\begin{align*}
    c_{3,4} &= 1 * 1 + -2 * -2 + 2 * -3 + -1 * -1 \\
    c_{3,4} &= 1 + 4 - 6 + 1 \\
    c_{3,4} &= 0  \\
\end{align*}

\begin{align*}
    c_{4,1} &= 1 * -1 + -1 * 2 + 1 * 2 + -1 * -1 \\
    c_{4,1} &= -1 - 2 + 2 + 1 \\
    c_{4,1} &= 0  \\
\end{align*}

\begin{align*}
    c_{4,1} &= 1 * -1 + -1 * 1 + 1 * 1 + -1 * -1 \\
    c_{4,2} &= -1 - 1 + 1 + 1 \\
    c_{4,2} &= 0  \\
\end{align*}

\begin{align*}
    c_{4,3} &= 1 * -1 + -1 * 1 + 1 * 2 + -1 * 0 \\
    c_{4,3} &= -1 - 1 + 2 + 0 \\
    c_{4,3} &= 0  \\
\end{align*}

\begin{align*}
    c_{4,4} &= 1 * 1 + -1 * -2 + 1 * -3 + -1 * -1 \\
    c_{4,4} &= 1 + 2 - 3 + 1 \\
    c_{4,4} &= 1  \\
\end{align*}

Therefore we can see that $c=I$ which is the identity matrix proving that $P_2^{-1}$ is indeed the inverse of $P_2$. \\

What are the coordinates over the basis consisting of the column vectors of $B_2$ of the vector whose coordinates over the basis  consisting of the column vectors of $A_2$ are $(2, -3, 0, 0)$?
\end{itemize}

To find these new coordinates, we can use the equation from the lectures where $U$ represents the old basis ($A$) and $V$ represents the new basis ($B$). We know the following equation: \\
\begin{align*}
    x_U &= P_{v,u}x_V \\
\end{align*}

We are given  $X_U$ and want to find $x_V$. Solving for $x_V$, we get: 
\begin{align*}
    x_V &= P_{V,U}^-1x_U \\
\end{align*}

We can substitute the given vector, $x_U$, and the inverse matrix to get:
\begin{align*}
    x_V &= \begin{pmatrix}
    -1 & -1 & -1 & 1\\
    2 & 1 & 1 & -2 \\
    2 & 1 & 2 & -3 \\
    -1 & -1 & 0 & -1
    \end{pmatrix} * \begin{pmatrix}
    2 \\
    -3 \\
    0 \\
    0 
    \end{pmatrix} \\
    \end{align*}

\begin{align*}
    x_{V_{1}} &= 2 * -1 + 2 * 2 + 2 * 2 + -1 * 2 \\
    x_{V_{1}} &= -2 + 4 + 4 - 2 \\
    x_{V_{1}} &= 2 \\
\end{align*}

\begin{align*}
    x_{V_{2}} &= -3 * -1 + -3 * 1 + -3 * 1 + -3 * -2 \\
    x_{V_{2}} &=  3 - 3 - 3 + 6\\
    x_{V_{2}} &= 3 \\
\end{align*}

\begin{align*}
    x_{V_{3}} &= 0 * -1 + 0 * 1 + 0 * 2 + 0 * 0 \\
    x_{V_{3}} &= 0 \\
\end{align*}

\begin{align*}
    x_{V_{4}} &= 0 * 1 + 0 * -2 + 0 * -3 + 0 * -1 \\
    x_{V_{4}} &= 0 \\
\end{align*}

So the new coordinates are $(2,3,0,0)$.
\section*{Problem 2: 30 points total}
\label{prob-3.3}
Consider the polynomials
\begin{align*}
B_0^2(t) & = (1 - t)^2  & B_1^2(t)  & = 2(1 - t)t & B_2^2(t) & = t^2 
&   &    \\
B_0^3(t) & = (1 - t)^3  & B_1^3(t) & = 3(1 - t)^2t & B_2^3(t) & = 3(1 - t)t^2 
 &  B_3^3(t) & = t^3,
\end{align*}
known as the {\it Bernstein polynomials\/} of degree $2$ and $3$.

\begin{itemize}
 \item[(1)](10 points)
Show that the Bernstein polynomials $B_0^2(t), B_1^2(t), B_2^2(t)$
are expressed as linear combinations of the basis
$(1, t, t^2)$ of the vector space of polynomials of degree at most $2$ 
as follows:
\[
\begin{pmatrix}
B_0^2(t)\\
B_1^2(t)\\
B_2^2(t)
\end{pmatrix} = 
\begin{pmatrix}
1 & -2  & 1 \\
0 &  2  & -2 \\
0 &  0  & 1  
\end{pmatrix} 
\begin{pmatrix}
1 \\
t \\
t^2
\end{pmatrix}. 
\]

Multiplying this out, we get that the matrix represented by the Bernstein polynomials is as follows: 
\begin{align*}
    \begin{pmatrix}
    B_0^2(t) = 1 * 1 + -2 * t + 1 * t^2 \\
    B_1^2(t) = 0 * 1 + 2 * t + -2 * t^2 \\
    B_2^2(t) = 0 * 1 + 0 * t + 1 * t^2 \\
    \end{pmatrix} &= \\
    \begin{pmatrix}
    B_0^2(t) = 1 + -2t + t^2 \\
    B_1^2(t) = 0 + 2t + -2t^2 \\
    B_2^2(t) = 0 + 0t + t^2 \\
    \end{pmatrix} &= \\
    \begin{pmatrix}
    B_0^2(t) = (1 - t)^2 \\
    B_1^2(t) = 2(1 - t)t \\
    B_2^2(t) = t^2 \\
    \end{pmatrix}
\end{align*}

Prove that
\[
B_0^2(t) +  B_1^2(t) +  B_2^2(t) = 1.
\] \\

Substituting the definitions of the Bernstein polynomials, we get:
\begin{align*}
    B_0^2(t) +  B_1^2(t) +  B_2^2(t) &= 1 \\
    (1 - t)^2 + 2(1 - t)t + t^2 &= 1 \\
    t^2 - 2t + 1 + 2t - 2t^2 + t^2 &= 1 \\
    1 &= 1 \\
\end{align*}
Therefore we can show that the Bernstein polynomials of degree 2 combine to equal 1.

\item[(2)](10 points)
Show that the Bernstein polynomials $B_0^3(t), B_1^3(t), B_2^3(t), B_3^3(t)$
are expressed as linear combinations of the basis
$(1, t, t^2, t^3)$ of the vector space of polynomials of degree at most $3$ 
as follows:
\[
\begin{pmatrix}
B_0^3(t)\\
B_1^3(t)\\
B_2^3(t) \\
B_3^3(t) 
\end{pmatrix} = 
\begin{pmatrix}
1 & -3  & 3 & -1 \\
0 &  3  & -6 & 3 \\
0 &  0  & 3  & -3 \\
0 &  0  & 0  & 1
\end{pmatrix} 
\begin{pmatrix}
1 \\
t \\
t^2\\
t^3
\end{pmatrix}. 
\]

Multiplying this out, we can solve for the Bernstein polynomials of degree 3 and check to see if they are equal to the definition of the Bernstein polynomials.

\begin{align*}
    \begin{pmatrix}
    B_0^3(t) = 1 * 1 + -3 * t + 3 * t^2 + -1 * t^3 \\
    B_1^3(t) = 0 * 1 + 3 * t + -6 * t^2 + 3 & t^3 \\
    B_2^3(t) = 0 * 1 + 0 * t + 3 * t^2 + -3 * t^3 \\
    B_3^3(t) = 0 * 1 + 0 * t + 0 * t^2 + 1 * t^3
    \end{pmatrix} &= \\
    \begin{pmatrix}
        B_0^3(t) = 1 - 3t + 3t^2 - t^3 \\
        B_1^3(t) = 0 + 3t - 6t^2 + 3t^3 \\
        B_2^3(t) = 0 + 0 + 3t^2 - 3t^3 \\
        B_3^3(t) = t^3
    \end{pmatrix} &= \\
    \begin{pmatrix}
        B_0^3(t) = -t^3 + 3t^2 - 3t + 1 \\
        B_1^3(t) = 3(1 - 2t + t^2)t \\
        B_2^3(t) = 3(1 - t)t^2 \\
        B_3^3(t) = t^3 
    \end{pmatrix} &= \\
    \begin{pmatrix}
        B_0^3(t) = (1 - t)^3 \\
        B_1^3(t) = 3(1 - t)^2t \\
        B_2^3(t) = 3(1 - t)t^2 \\
        B_3^3(t) = t^3
    \end{pmatrix}
\end{align*}

Prove that
\[
B_0^3(t) +  B_1^3(t) +  B_2^3(t) + B_3^3(t) = 1.
\]

Substituting the definition for the Bernstein polynomials of degree 3 we get:
\begin{align*}
    B_0^3(t) +  B_1^3(t) +  B_2^3(t) + B_3^3(t) &= 1 \\
    (1 - t)^3 + 3(1 - t)^2t + 3(1 - t)t^2 + t^3 &= 1 \\
    -t^3 + 3t^2 - 3t + 1 + 3t - 6t^2 + 3t^3+ 3t^2 - 3t^3 + t^3 &= 1 \\
    1 &= 1 \\
\end{align*}

\item[(3)](10 points)
Prove that the Bernstein polynomials of degree $2$
are linearly independent, and that
the Bernstein polynomials of degree $3$
are linearly independent.
 \end{itemize}

Bernstein Polynomials of degree 2 proof: \\
Suppose for the sake of contradiction that 

\section*{Problem 3: 10 points}
\label{prob-5.2}
Prove that 
for every vector space $E$, if $\mapdef{f}{E}{E}$ is an idempotent
linear map, i.e., $f\circ f = f$, then we have a direct sum
\[
E = \Ker{f} \oplus \Im{f},
\]
so that $f$ is the projection onto its image $\Im{f}$.


\section*{Problem 4: 20 points plus 15 points Extra Credit}
\label{prob-5.5}
Given any vector space $E$, a linear map  $\mapdef{f}{E}{E}$  is an
{\it involution\/} if  $f\circ f = \id$.

\begin{itemize}
\item[(1)](10 points)
Prove that an  involution $f$ is invertible. What is its inverse?

\item[(2)](10 points)
Let $E_1$ and $E_{-1}$ be the subspaces of $E$ defined as follows:
\begin{align*}
E_1 & = \{u \in E \mid f(u) = u\} \\
E_{-1} & = \{u \in E \mid f(u) = -u\}.
\end{align*}
Prove that we have a direct sum
\[
E = E_{1} \oplus E_{-1}.
\]

\hint
For every $u\in E$, write
\[
u = \frac{u + f(u)}{2} + \frac{u - f(u)}{2}. 
\]

\item[(3)]{ \bf Extra credit }(15 points)
If $E$ is finite-dimensional and $f$ is an involution, prove that
there is some basis of $E$ with respect to which the matrix of $f$ is of the form
\[
I_{k , n - k} =
\begin{pmatrix}
I_k & 0 \\
0 & - I_{n - k}
\end{pmatrix},
\]
where $I_k$ is the $k\times k$ identity matrix 
(similarly for $I_{n -  k}$) and $k = \mathrm{dim}(E_1)$.
Can you give a geometric interpretation of the action of $f$
(especially when   $k = n - 1$)?

\section*{Total: 70 points\qquad Extra Credit: 15 points}
\end{itemize}


\end{document}